\section{Introduction}

One of the myriad uses of ocean models is in developing ocean heat uptake estimates and overturning circulation predictions \citep{armour16}. Additionally, the overturning circulation itself affects the wider climate, which manifests when ocean models are used as a component of coupled climate simulations. The problems of ocean heat uptake and overturning circulation are both strongly defined by the density structure of the ocean, which is modified by mixing. For example, mixing at depth controls the abyssal overturning cell that constitutes part of the meridional overturning circulation \citep{mashayek15}. As a result, models may be unable to accurately constrain the abyssal overturning if the magnitude of spurious diapycnal mixing cannot be completely controlled.

- define "spurious diapycnal mixing"
- Cat's paper says surface mixing is important too (wrt "mixing at depth")

Numerical ocean models are governed by approximations of the incompressible Navier-Stokes equations for momentum, also known as the primitive equations \citep{griffies04}. In these models, the vertical balance is hydrostatic, where the vertical pressure gradient force is balanced by the gravitational force. The mixing of momentum by the unresolved eddy field is parameterised by an explicit eddy viscosity term. Potential density of water parcels is controlled by salinity and potential temperature through an equation of state. These tracers are advected by the explicitly resolved eddy field, and mixed by the unresolved eddy field through a parameterised eddy diffusion term.

To solve the primitive equations, ocean models implement some kind of discretisation, such as the finite volume method. This discretisation involves representing the computational domain as a series of grid cells in three-dimensional space, where each grid cell has associated mean velocities and tracer concentrations. Horizontal tracer advection schemes are discretisations of the advection equation that create higher-order reconstructions of the tracer field using information from neighbouring grid cells. Often, tracer advection schemes are coupled with a flux limiter, which prevents the creation of spurious minima or maxima in tracer concentration.

Mixing processes create fluxes of tracer between grid cells. In ocean models, mixing has two main causes, physical and numerical. A small fraction of physical mixing comes from molecular diffusion. The rest comes from advection by numerically unresolved eddies, which is parameterised as a diffusive process. On the other hand, numerical mixing arises from the discretisations and algorithms used by the ocean model in implementing the governing equations. Numerical mixing is also known as spurious mixing and has no physical basis. For example, first-order upwind advection has numerical diffusion as the leading error term \citep{gentry66}.

- models don't implement molecular diffusion, that's parameterised

Spurious mixing is undesirable in ocean models. It is unphysical, it may add to the imposed and parameterised mixing to an unknown extent and it is difficult to diagnose. Spurious mixing affects numerical experiments which are contingent on the density structure of the ocean. Ocean heat uptake or overturning circulation strength in such experiments may be biased. One of the considerations in model development and configuration is thus to ensure spurious mixing is minimised.

The magnitude of spurious mixing is strongly controlled by the choice of horizontal tracer advection scheme. Much of the focus in reducing spurious mixing has therefore been on tracer advection, through improving numerical accuracy or the model's subgrid scale representations. Some argue that a high-order advection scheme is sufficient to reduce the spurious mixing to acceptable levels \citep{daru04}. This is simply a matter of using a sufficiently high-order polynomial reconstruction to try to capture the overall structure of tracer distributions. Other advection schemes attempt to preserve the subgrid scale representation of a given field \citep{prather86}. By carrying information about both first and second-order moments, the model is able to exactly reconstruct a field to second order. The second-order moment scheme must often be used in conjunction with a flux limiter to ensure against the creation of spurious minima and maxima, which in essence reduces back to a first-order advection scheme and so no longer preserves second-order moments when the limiter is active. An alternative view is that the tracer advection scheme only needs sufficient accuracy before grid-scale noise in velocity becomes the dominant source of spurious mixing \citep{ilicak12}. There are further considerations in model configuration beyond tracer advection, such as the vertical coordinate.

With an open choice of vertical coordinate, it's not clear which is the ideal choice for a specific class of modelling. The terrain-following sigma coordinate is often used for coastal modelling, but may present issues with pressure gradient calculation due to strongly sloping coordinate surfaces. To keep the advantages of a terrain-following coordinate, but reduce pressure gradient errors and spurious mixing, \citet{hofmeister10} formulated an adaptive terrain-following grid. Vertical layer positions are modified through a vertical diffusion proportional to shear, stratification and distance from boundaries, whereas the grid is smoothed in the horizontal. Another adaptive vertical grid is z-tilde \citep{leclair11}, which has Lagrangian behaviour to motions on short timescales, but relaxes to a target grid over long timescales to prevent the grid from drifting. This scheme is good for allowing the propagation of internal gravity waves. A final example in the grid used by the HyCOM model \citep{bleck02}, which adapts to different coordinates depending on location, such as terrain-following near boundaries, or isopycnal at depth. In isolation, each coordinate has strengths and weaknesses for ocean modelling, but the combination attempts to preserve the strengths of each.

To allow generalised vertical coordinates, models can make use of an arbitrary Lagrangian-Eulerian (ALE) scheme. There are two general implementations of ALE in ocean models, depending on the reference frame of the model \citep{margolin03, leclair11}. In quasi-Eulerian models, any changes in the vertical grid due to the choice of coordinate are incorporated into the solution of the primitive equations \citep{kasahara74}. This is often done by calculating the motion of the new vertical grid relative to the old grid as a vertical velocity. As such, there's an associated spurious mixing with advection in both the horizontal and vertical directions.

The quasi-Lagrangian algorithm \citep{hirt74} is for models which are implemented in a Lagrangian frame of reference. Here, the vertical grid may move during the solution of the primitive equations or as a consequence of parameterisations such as Gent-McWilliams thickness diffusion \citep{gent90}. This is followed by the regridding phase, where the new vertical grid is calculated. Finally, the new grid is applied in the remapping phase, during which the model state is mapped onto the new grid. The remapping algorithm is often an adaptation of an advection scheme \citep{margolin03}. Spurious mixing may occur during the remapping phase, depending on both the new grid and the subgrid scale reconstruction of tracers on the old grid. We will focus on the quasi-Lagrangian algorithm, since it is used by MOM6, the model we are evaluating in this paper.

There have been some investigations into the accuracy of ALE in ocean models. \citet{white08} demonstrated the development and implementation of an accurate reconstruction scheme for the remapping stage of ALE, with their piecewise quartic method (PQM). PQM is the most accurate reconstruction method present in MOM6, and was shown to significantly increase reconstruction accuracy for a small increase in computational cost. However, the actual performance of PQM was not considered in a model. The impacts of different reconstruction schemes in regridding and remapping were considered by \citet{white09}, comparing their spurious mixing in terms of the change of volume distributions across density classes. Neither of these studies quantified the magnitude of spurious mixing in total, or as a comparison to the spurious mixing by advection. Formulating this comparison is one of the aims of this paper.

In attempting to evaluate the performance of numerical schemes with regard to spurious mixing, there is no consensus on the diagnostic technique to use. \citet{griffies00} used an effective diapycnal diffusivity, which allows for direct comparison between the spurious mixing and expected oceanic values. However, because it uses a reference density profile compiled from the entire domain, the effective diffusivity is only a single idealised vertical profile, and can't be mapped back to real space in any meaningful manner.

An alternative to diagnosing spurious mixing from the model state is to calculate an analytical solution from the advection operator itself. \citet{moralesmaqueda06} did this with upstream based schemes, such as the second-order moment method of Prather, calculating a closed form expression for the implicit numerical diffusivity.

Substituting the second-order moment scheme for an arbitrary choice of horizontal advection scheme, \citet{burchard08} showed that by considering the destruction of variance of a tracer by horizontal advection, the impact on subgrid scale structure can be inferred. This leads to a general diagnostic which gives a comparison of physical and numerical mixing through subgrid scale changes. Tracer variance can be calculated for every model gridpoint, and thus the variance destruction gives information about the relative impact of physical and numerical mixing through full space, given a statistically significant integration period.

A simpler diagnostic of spurious mixing is simply to observe its effects on the reference potential energy RPE, \citep{winters95}. This gives only timeseries data (no localised information), but allows for ready comparison across models for the same physical configuration. \citet{ilicak12} used the rate of change of RPE in analysing the role of momentum closure between different models (GOLD, MITGCM, MOM and ROMS). Comparisons were performed across a suite of test cases intended to stress different physical phenomena: a lock exchange, downslope flow, internal gravity waves, baroclinic eddies, and a global spindown.

\citet{ilicak12} studied the dependence on the lateral grid Reynolds number

$$\mathrm{Re}_\Delta = \frac{U\Delta x}{\nu_h},$$

where $U$ is the characteristic horizontal velocity scale, $\Delta x$ is the horizontal grid spacing and $\nu_h$ is the horizontal viscosity coefficient. For the dissipation of spurious grid-scale noise in the velocity field, the lateral grid Reynolds number should be less than 2 \citep[p.~410]{griffies04}. By varying the horizontal viscosity, \citet{ilicak12} showed that spurious mixing increases with the lateral grid Reynolds number up until saturation at approximately $\mathrm{Re}_\Delta = 10$. This demonstrates that the momentum closure must be chosen such that it reduces spurious grid-scale noise, which causes a saturation in the spurious mixing.

- saturation due to flux limiter, otherwise gridscale noise would increase with grid Re?

To look at the performance of a model with an ALE scheme, \citet{petersen15} extended the study of Ilicak et al. In addition to the z-star and isopycnal coordinates, three additional vertical coordinates were used to demonstrate the ALE in the MPAS-Ocean (hereafter referred to as MPAS-O) model: the terrain-following sigma coordinate, z-level, and z-tilde \citep{leclair11}. To compare to another model with a z-level vertical coordinate, POP was also added to the suite of models.

As MPAS-O is a quasi-Eulerian model, there is a resolved transport across vertical layer interfaces during tracer advection. Use of z-tilde leads to a reduction in this transport, and therefore a reduction in spurious diapycnal mixing associated with the choice of vertical coordinate. However, the model timestep had to be halved in global simulations with z-tilde, which has a significant impact on computational cost. Additionally, the z-tilde coordinate was shown to be unsuitable for simulations with large, transient flows, highlighting the need for further development and evaluation of vertical coordinates in models.

This paper has two main aims. Firstly, to evaluate the performance of another ALE model, MOM6, against the models exhibited by \citet{ilicak12} and \citet{petersen15}. The comparison is made using both the standard configurations, and with a coordinate that is unique to MOM6, continuous isopycnal. Secondly, a method is proposed for using RPE changes to separate the contributions of horizontal and vertical processes (i.e. advection and ALE). This method allows for the evaluation of different advection schemes, and different orders of reconstruction in ALE, and is proposed as a useful tool in comparing between different vertical coordinates.
